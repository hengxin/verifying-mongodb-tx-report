% checking.tex

%%%%%%%%%%%%%%%%%%%%
\begin{frame}{}
  \fig{width = 0.50\textwidth}{figs/checking}

  \begin{center}
	多项式时间 SI 变体白盒检测算法
  \end{center}
\end{frame}
%%%%%%%%%%%%%%%%%%%%

%%%%%%%%%%%%%%%%%%%%
\begin{frame}{}
  \fig{width = 0.30\textwidth}{figs/basic-idea}

  \begin{enumerate}[(1)]
	\setlength{\itemsep}{6pt}
	\item 收集事务协议产生的历史执行
	\item 根据证明中的定义构造 $\vis$ 与 $\ar$ 关系
	\item 检查 $\vis$ 与 $\ar$ 关系是否满足相应的 $\si$ 变体所需的公理
  \end{enumerate}

  \pause
  \vspace{0.20cm}
  \begin{center}
	第 (2) 中构造的 $\ar$ 是全序关系, 提供了 Version Order, \\[5pt]
	第 (3) 步可以在多项式时间内完成
  \end{center}
\end{frame}
%%%%%%%%%%%%%%%%%%%%

%%%%%%%%%%%%%%%%%%%%
\begin{frame}{}
  \begin{center}
	如何从历史执行中获取必要的信息, 以构造 $\vis$ 与 $\ar$ 关系:
  \end{center}

  \vspace{0.30cm}
  \begin{description}[$\scalg:$]
	\setlength{\itemsep}{10pt}
	\item[$\wtalg:$] 记录事务的开始与提交时间, 利用前述引理校正
	\item[$\rsalg:$] 分别从 \texttt{mongod.log} 与 \texttt{oplog.rs} 中获取 \\[5pt]
	  事务的读时间戳与提交时间戳
	\item[$\scalg:$] 分别从 \texttt{mongod.log} 与分区主节点 \oplog{} 中获取 \\[5pt]
	  事务的读时间戳与提交时间戳
  \end{description}

  \vspace{0.20cm}
  \begin{lemma}[冲突事务的提交顺序]
	\vspace{-0.30cm}
    \begin{align*}
      &\forall \txnvar, \txnvar' \in \WTTXN.\; \\
    	  &\quad \txnvar \conflict \txnvar' \implies (\txnvar \rel{\arwt} \txnvar'
    	  \iff \txnvar.\tid < \txnvar'.\tid).
    \end{align*}
  \end{lemma}
\end{frame}
%%%%%%%%%%%%%%%%%%%%

%%%%%%%%%%%%%%%%%%%%
\begin{frame}{}
  % table-deployment.tex

\begin{table}[H]
  \centering
  \renewcommand{\arraystretch}{2.0}
  \resizebox{1.00\textwidth}{!}{%
  \begin{tabular}{|c|c|c|c|}
      \hline
      \textbf{Deployment} & \textbf{Version}
          & \textbf{Configuration} & \textbf{OS} \\ \hline
      Standalone          & WiredTiger 3.3.0
          & A WiredTiger storage engine   & Ubuntu 20.04  \\ \hline
      Replica Set         & \multirow{2}{*}{MongoDB 4.2.8}
        & A replica set of 5 nodes
        & \multirow{2}{*}{\begin{tabular}[c]{@{}c@{}}
          \\ Each database node runs in a Docker \\
          container built with the Debian 10 image.\end{tabular}} \\ \cline{1-1} \cline{3-3}
      Sharded Cluster     &
        & \begin{tabular}[c]{@{}c@{}}A cluster consists of 1 config server and 2 shards. \\
          Each shard is a replica set of 3 nodes. \\
          Each node has a mongos instance.\end{tabular}
        & \\ \hline
  \end{tabular}}
\end{table}
  \begin{center}
	MongoDB 部署配置
  \end{center}
\end{frame}
%%%%%%%%%%%%%%%%%%%%

%%%%%%%%%%%%%%%%%%%%
\begin{frame}{}
  % table-generator.tex

\begin{table}[H]
  \centering
  \renewcommand{\arraystretch}{2.0}
  \resizebox{1.00\textwidth}{!}{%
  \begin{tabular}{|l|l|l|}
    \hline
    \textbf{Parameter} & \textbf{Value(s)} & \textbf{Description}                                                                                                \\ \hline
    key-count          & 10  & There are 10 distinct keys at any point for generation.                                                              \\ \hline
    key-dist           & exponential  & Probability distribution on keys. \\ \hline
    max-txn-length      & \set{4, 8, 12}   & Each transaction contains at most 4/8/12 operations.                                                                    \\ \hline
    max-updates-per-key & 128      & There are at most 128 updates on each key. \\ \hline
    read:update ratio  & 1 : 1 & The default (and fixed) read:update ration in Jepsen. \\ \hline
    wt-duration           & \set{10s, 20s, 30s, 40s, 50s, 60s} & The duration of an experiment on WiredTiger. \\ \hline
    rs-duration           & \set{1min, 2min, 3min, 4min, 5min} & The duration of an experiment on replica set . \\ \hline
    sc-duration           & \set{1min, 2min, 3min, 4min, 5min, 8min} & The duration of an experiment on sharded cluster. \\ \hline
    wt/rs/sc-timeout & 5s/10s/30s & \begin{tabular}[c]{@{}c@{}}Transaction timeout for experiments on \\
        WiredTiger, replica set, and sharded cluster, respectively.\end{tabular} \\ \hline
  \end{tabular}}
\end{table}
  \begin{center}
	事务生成器相关参数 (Jepsen 支持)
  \end{center}
\end{frame}
%%%%%%%%%%%%%%%%%%%%

%%%%%%%%%%%%%%%%%%%%
\begin{frame}{}
  \fig{width = 1.00\textwidth}{figs/check-mongo-perf}

  \begin{itemize}
	\setlength{\itemsep}{10pt}
	\item 13ms 的时钟误差内, 可认为 $\wtalg \models \strongsi$
	\item 10s 检测包含 30, 000 个事务的历史执行 \\[5pt]
	  \begin{itemize}
		\item Intel Core i5-9500 CPU @ 3.00GHZ and 16GB RAM
	  \end{itemize}
  \end{itemize}
\end{frame}
%%%%%%%%%%%%%%%%%%%%