% rsalg.tex

%%%%%%%%%%%%%%%%%%%%
\begin{frame}{}
  \fig{width = 0.50\textwidth}{figs/mongodb-replica-set}

  \[
	\rsalg \models \rtsi
  \]
\end{frame}
%%%%%%%%%%%%%%%%%%%%

%%%%%%%%%%%%%%%%%%%%
\begin{frame}{}
  \begin{center}
	\rc = ``\snapshotrc'' and \wc = ``\majority''
  \end{center}

  \vspace{0.30cm}
  \begin{description}
	\setlength{\itemsep}{10pt}
	\item[\rc = ``\snapshotrc'':] 保证事务读取到一致性的、被多数节点 \\[5pt] 提交的快照
	\item[\wc = ``\majority'':] 保证事务中的写操作以及读到的数据 \\[5pt]
	  被多数节点提交 (majority committed)
  \end{description}
\end{frame}
%%%%%%%%%%%%%%%%%%%%

%%%%%%%%%%%%%%%%%%%%
\begin{frame}{}
  \begin{center}
	主节点维护 \oplog, 并负责决定事务的提交顺序

	\vspace{0.50cm}
	\fig{width = 0.50\textwidth}{figs/oplog}

	\pause
	\vspace{0.50cm}
	每个事务 $\txnvar$ 被赋予一个唯一的逻辑提交时间戳 $\txnvar.\committs$

	\vspace{0.50cm}
	这些逻辑时间戳确定了事务在 ReplicaSet 层的(逻辑)提交顺序
  \end{center}
\end{frame}
%%%%%%%%%%%%%%%%%%%%

%%%%%%%%%%%%%%%%%%%%
\begin{frame}{}
  \begin{center}
	当事务 $\txnvar$ 开始时 ($\openwtsession$), \\[5pt]
	计算它的``读时间戳'' ($\txnvar.\readts$)

	\vspace{0.50cm}
	\fig{width = 0.50\textwidth}{figs/oplog}

	\vspace{0.50cm}
	条件: \oplog{} 中所有提交时间戳小于 $\txnvar.\readts$ 的事务均已提交

	\vspace{0.30cm}
	($\txnvar.\readts$ 保证可见的事务在 \oplog{} 中不造成空洞)
  \end{center}
\end{frame}
%%%%%%%%%%%%%%%%%%%%

%%%%%%%%%%%%%%%%%%%%
\begin{frame}{}
  \begin{center}
	\fig{width = 0.70\textwidth}{figs/SpeculativeSI}

	\vspace{0.30cm}
	\begin{itemize}
	%   \setlength{\itemsep}{10pt}
	  \item 事务读取 WiredTiger 层当前最新的数据, 而不限于已被多数节点提交的数据
	  \item 当事务提交时, \emph{等待}读取到的数据被多数节点提交
	\end{itemize}

	\pause
	只读事务在提交时发起``\noop''操作, 然后等待该操作被多数节点提交
  \end{center}
\end{frame}
%%%%%%%%%%%%%%%%%%%%

%%%%%%%%%%%%%%%%%%%%
\begin{frame}{}
  \fig{width = 0.90\textwidth}{figs/hlc}

  \begin{columns}
	\column{0.20\textwidth}
	\column{0.60\textwidth}
	  % hlc.tex

\begin{algorithm}[H]
  \begin{algorithmic}[1]
    % \Statex Hybrid logical clocks are compared lexicographically:
    \Procedure{\tick}{\null}
      \label{line:procedure-tick}
      \If{$\ct.\sec \geq \clock$}
        \label{line:tick-sec-clock}
        \State \Return $\langle \ct.\sec, \ct.\counter + 1 \rangle$
        \label{line:tick-counter}
      \Else
        \State \Return $\langle \clock, 0 \rangle$
        \label{line:tick-clock}
      \EndIf
    \EndProcedure
  \end{algorithmic}
\end{algorithm}
	\column{0.20\textwidth}
  \end{columns}

  \pause
  \vspace{0.30cm}
  \begin{center}
	\begin{itemize}
	  \setlength{\itemsep}{6pt}
	  \item 主节点的 cluster time ($\ct$) 仅在产生新的 \oplog{} 项时增加
	  \item 每个消息都携带发送方的 $\ct$
	  \item 每个节点(包括客户端)维护它已知的最大 $\ct$
	\end{itemize}
  \end{center}
\end{frame}
%%%%%%%%%%%%%%%%%%%%

%%%%%%%%%%%%%%%%%%%%
\begin{frame}{}
  visibility
\end{frame}
%%%%%%%%%%%%%%%%%%%%

%%%%%%%%%%%%%%%%%%%%
\begin{frame}{}
  \begin{definition}[Visibility Relation]
	\vspace{-0.30cm}
	\begin{align*}
      &\forall \txnvar_{1}, \txnvar_{2} \in \RSTXN.\;
        \txnvar_{1} \rel{\visrs} \txnvar_{2} \iff \\
        &\qquad \txnvar_{1}.\committs \le \txnvar_{2}.\readts.
	\end{align*}
  \end{definition}
\end{frame}
%%%%%%%%%%%%%%%%%%%%

%%%%%%%%%%%%%%%%%%%%
\begin{frame}{}
  \begin{definition}[Arbitration Relation]
	Order all update transactions in $\arrs$ by their $\committs$.
    For each client, we insert its read-only transactions
    one by one in session order:
    each read-only transaction $\txnvar$
    on session $\rssessionvar$ is placed immediately
    after the later (in $\arrs$) of
    \begin{enumerate}[(1)]
      \item the previous (in $\sors$) transaction of $\txnvar$
        on session $\rssessionvar$, if any; and
      \item the update transactions $\txnvar'$ (in $\arrs$)
        such that $\txnvar' \rel{\visrs} \txnvar$.
    \end{enumerate}
  \end{definition}
\end{frame}
%%%%%%%%%%%%%%%%%%%%

%%%%%%%%%%%%%%%%%%%%
\begin{frame}{}
\end{frame}
%%%%%%%%%%%%%%%%%%%%