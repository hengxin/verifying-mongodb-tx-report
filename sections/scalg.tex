% scalg.tex

%%%%%%%%%%%%%%%%%%%%
\begin{frame}{}
  \fig{width = 0.55\textwidth}{figs/mongodb-shard}

  \[
  	\scalg \models \sessionsi
  \]
\end{frame}
%%%%%%%%%%%%%%%%%%%%

%%%%%%%%%%%%%%%%%%%%
\begin{frame}{}
  \begin{center}
    分布式事务

    \vspace{0.60cm}
    不同分区间使用 2PC 协议, 保证原子性

    \vspace{0.30cm}
    同一分区间使用共识复制协议, 提高容错性
  \end{center}
\end{frame}
%%%%%%%%%%%%%%%%%%%%

%%%%%%%%%%%%%%%%%%%%
\begin{frame}{}
  \fig{width = 1.00\textwidth}{figs/2pc-trim}
\end{frame}
%%%%%%%%%%%%%%%%%%%%

%%%%%%%%%%%%%%%%%%%%
\begin{frame}{}
  \begin{center}
    monogs 为事务指定快照时间戳 (\readts)

    \vspace{0.50cm}
    逻辑时间戳是松散同步的, 在某些分区上该快照尚不可得
  \end{center}

  \pause
  \vspace{0.50cm}
  \begin{description}
    \setlength{\itemsep}{10pt}
    \item[\caseclockskew:] 如果 $\ct < \readts$, 则先推进 $\ct$
    \item[\casependingcommitread:] 如果可见的写操作是 \\[5pt] \wtprepareinprogress{} 状态, 则重试读
    \item[\casependingcommitupdate] 如果可见的写操作是 \\[5pt] \wtprepareinprogress{} 状态, 则先重试读
    \item[\caseholes:] 等待直到 \oplog{} 中 $\committs \le \readts$ 的事务均提交
  \end{description}
\end{frame}
%%%%%%%%%%%%%%%%%%%%

%%%%%%%%%%%%%%%%%%%%
\begin{frame}{}
  \begin{definition}[可见关系 $\vissc$]
    \vspace{-0.30cm}
    \begin{align*}
      &\forall \txnvar_{1}, \txnvar_{2} \in \SCTXN.\;
        \txnvar_{1} \rel{\vissc} \txnvar_{2} \iff \\
        &\qquad \txnvar_{1}.\committs \le \txnvar_{2}.\readts.
    \end{align*}
  \end{definition}
\end{frame}
%%%%%%%%%%%%%%%%%%%%

%%%%%%%%%%%%%%%%%%%%
\begin{frame}{}
  \begin{definition}[仲裁关系 $\arsc$]
	  \begin{itemize}
	    \setlength{\itemsep}{10pt}
	    \item 非只读事务按照它们的 $\committs$ 在 $\arsc$ 中排序 \\[5pt]
        \begin{itemize}
          \item 相同的 $\committs$ 根据事务协调者所在的分区 $\shardid$ 定序
        \end{itemize}
	    \item 对于每个客户端, 将只读事务按照会话序依次插入 $\arsc$。\\[5pt]
	      具体而言, 会话 $\scsessionvar$ 上的只读事务 $\txnvar$ 紧随以下事务之后: \\[5pt]
	      \begin{itemize}
	  	    \setlength{\itemsep}{5pt}
          \item $\txnvar$ 在 $\sosc$ 序下的前一个事务 (如果有),
          \item 非只读事务 $\txnvar'$ 满足 $\txnvar' \rel{\vissc} \txnvar$。
        \end{itemize}
    \end{itemize}
  \end{definition}
\end{frame}
%%%%%%%%%%%%%%%%%%%%

%%%%%%%%%%%%%%%%%%%%
\begin{frame}{}
  \[
  	\scalg \models \sessionsi
  \]
  \fig{width = 0.55\textwidth}{figs/mongodb-shard}
  \[
  	\sosc \subseteq \vissc
  \]
  \begin{center}
    由节点维护并分发逻辑时间戳的机制保障
  \end{center}
\end{frame}
%%%%%%%%%%%%%%%%%%%%