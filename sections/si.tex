% si.tex

%%%%%%%%%%%%%%%%%%%%
\begin{frame}{}
  \begin{columns}
	\column{0.20\textwidth}
	\column{0.60\textwidth}
	  \begin{enumerate}
	  	\setlength{\itemsep}{12pt}
	  	\item 事务 $T:$ $(E, \po)$
	  	  \begin{itemize}
	  	    \setlength{\itemsep}{2pt}
	  	    \item $\po:$ Program Order
	  	    \item $\starttime(T):$ 事务开始时间
	  	    \item $\committime(T):$ 事务提交时间
	  	  \end{itemize}
	  	\item 历史 $\h:$ $(\T, \so)$
	  	  \begin{itemize}
	  	    \setlength{\itemsep}{2pt}
			\item $\T:$ 已提交事务集合
	  	  	\item $\so:$ Session Order
	  	  \end{itemize}
	  	\item 执行 $\ae:$ $(\h, \vis, \ar)$
	  	  \begin{itemize}
	  	  	\setlength{\itemsep}{2pt}
	  	  	\item $\vis:$ 可见性 (Visibility) 偏序关系
	  	  	\item $\ar:$ 仲裁 (Arbitration) 全序关系
	  	  	\item $\vis \subseteq \ar$
	  	  \end{itemize}
	  \end{enumerate}
	\column{0.20\textwidth}
  \end{columns}
\end{frame}
%%%%%%%%%%%%%%%%%%%%

%%%%%%%%%%%%%%%%%%%%
\begin{frame}{}
  \begin{center}
	一个事务一致性模型可定义为一组一致性公理的集合 $\Phi$。

	\vspace{0.80cm}
	历史 $\h$ 满足事务一致性模型 $\Phi$,
	如果存在 $\vis$ 与 $\ar$ 使得
	\[
	  \exists \vis, \ar.\; (\h, \vis, \ar) \models \Phi\text{。}
	\]
  \end{center}
\end{frame}
%%%%%%%%%%%%%%%%%%%%

%%%%%%%%%%%%%%%%%%%%
\begin{frame}{}
  % table-axioms.tex

\begin{table}[H]
  \centering
  \renewcommand{\arraystretch}{2.0}
  \resizebox{1.00\textwidth}{!}{%
    \begin{tabular}{|c|c|}
    \hline
    \multicolumn{2}{|c|}{
	  $\begin{aligned}[c]
		&\forall (\E, \po) \in \h.\;
		  \forall \evar \in \Event.\;
		    \forall \keyvar, \valvar.\;
			  \big(\op(\evar) = \readevent(\keyvar, \valvar)
			  \land \set{\fvar \mid (\op(\fvar) = \_(\keyvar, \_) \land \fvar \rel{\po} \evar} \neq \emptyset\big) \\
			  &\quad \implies \op(\max_{\po} \set{\fvar \mid \op(\fvar) = \_(\keyvar, \_) \land \fvar \rel{\po} \evar}) = \_(\keyvar, \valvar)
	   \end{aligned}$ \hfill (\intaxiom)} \\ \hline
    \multicolumn{2}{|c|}{
	  $\begin{aligned}[c]
		\forall T \in \h.\; \forall \keyvar, \valvar.\; T \vdash \readevent(\keyvar, \valvar) \implies
		  \max_{\ar}(\vis^{-1}(T) \cap \WriteTx_{\keyvar}) \vdash \writeevent(\keyvar, \valvar)
	  \end{aligned}$ \hfill (\extaxiom)} \\ \hline
    $\so \subseteq \vis$ \hfill (\sessionaxiom)
	& $\ar \comp \vis \subseteq \vis$ \hfill (\prefixaxiom) \\ \hline
	  $\rb \subseteq \vis$ \hfill (\rbaxiom)
	& $\cb \subseteq \ar$ \hfill (\cbaxiom) \\ \hline
	  $\vis \subseteq \rb$ \hfill (\realtimesnapshotaxiom)
	& $\forall S, T \in \h.\; S \conflict T \implies (S \rel{\vis} T \lor T \rel{\vis} S)$
	  \hfill (\noconflictaxiom) \\ \hline
    \end{tabular}%
  }
\end{table}
    %   $\forall S, T \in \h.\; S \rel{\vis} T \implies \starttime(T) > \committime(S)$
\end{frame}
%%%%%%%%%%%%%%%%%%%%

%%%%%%%%%%%%%%%%%%%%
\begin{frame}{}
  \[
	\si = \intaxiom \land \extaxiom \land \prefixaxiom \land \noconflictaxiom
  \]
\end{frame}
%%%%%%%%%%%%%%%%%%%%

%%%%%%%%%%%%%%%%%%%%
\begin{frame}{}
  \[
	\sessionsi = \si \land \sessionaxiom
  \]

  \pause
  \[
	\rtsi = \si \land \rbaxiom \land \cbaxiom
  \]

  \pause
  \[
	\gsi = \si \land \rtsi \land \cbaxiom
  \]

  \pause
  \[
	\strongsi = \gsi \land \rbaxiom
  \]
\end{frame}
%%%%%%%%%%%%%%%%%%%%

%%%%%%%%%%%%%%%%%%%%
\begin{frame}{}
  \begin{columns}
	\column{0.10\textwidth}
	\column{0.40\textwidth}
	  \begin{itemize}
	  	\item \ansisi
	  	\item \si
	  	\item \gsi
	  	\item \strongsi
	  	\item \strongsessionsi
	  	\item \parallelsi
	  	\item \wsi
	  	\item \nmsi
	  	\item \pcsi
	  \end{itemize}
	\column{0.50\textwidth}
	  \fig{width = 0.70\textwidth}{figs/SI-variants}
  \end{columns}
\end{frame}
%%%%%%%%%%%%%%%%%%%%

%%%%%%%%%%%%%%%%%%%%
\begin{frame}{}
  \begin{center}
	\fig{width = 0.80\textwidth}{figs/mongo-txn-si}

	\vspace{0.20cm}
	重点在于如何确定每个事务的``读快照'' (Read Snapshot), \\[5pt]
	也就是对该事务可见的所有事务构成的集合
  \end{center}
\end{frame}
%%%%%%%%%%%%%%%%%%%%