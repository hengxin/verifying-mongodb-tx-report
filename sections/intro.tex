% intro.tex

%%%%%%%%%%%%%%%%%%%%
\begin{frame}{}
  \fig{width = 0.70\textwidth}{figs/mongodb-transaction-logo}
\end{frame}
%%%%%%%%%%%%%%%%%%%%

%%%%%%%%%%%%%%%%%%%%
\begin{frame}{}
%   \fig{width = 1.00\textwidth}{figs/mongodb-txn}
  \begin{center}
	MongoDB 的三种经典部署架构
  \end{center}
\end{frame}
%%%%%%%%%%%%%%%%%%%%

%%%%%%%%%%%%%%%%%%%%
\begin{frame}{}
  \fig{width = 1.00\textwidth}{figs/mongodb-history}
  \begin{center}
	MongoDB 事务的三阶段发展过程
  \end{center}
\end{frame}
%%%%%%%%%%%%%%%%%%%%

%%%%%%%%%%%%%%%%%%%%
\begin{frame}{}
  \begin{center}
	\emph{\Large A Fundamental Question:} \\[20pt]

	\begin{quote}
	  {\large What transactional consistency guarantee do MongoDB transactions
	  in each deployment provide?}
	\end{quote}
  \end{center}
\end{frame}
%%%%%%%%%%%%%%%%%%%%

%%%%%%%%%%%%%%%%%%%%
\begin{frame}{}
  \begin{center}
	挑战一: MongoDB 官方规约不清楚, SI 有多种变体
  \end{center}

  \begin{columns}
	\column{0.50\textwidth}
	  \fig{width = 1.00\textwidth}{figs/SpeculativeSI}
	\column{0.50\textwidth}
	  \fig{width = 0.60\textwidth}{figs/SI-variants}
  \end{columns}
\end{frame}
%%%%%%%%%%%%%%%%%%%%

%%%%%%%%%%%%%%%%%%%%
\begin{frame}{}
  \begin{center}
	挑战二: MongoDB 缺少精简的事务协议描述, 更没有严格证明
  \end{center}

  \fig{width = 0.80\textwidth}{figs/mongodb-txn-spec}
\end{frame}
%%%%%%%%%%%%%%%%%%%%

%%%%%%%%%%%%%%%%%%%%
\begin{frame}{}
  \begin{center}
	挑战三: SI 检测问题是 \textsf{NP-complete} 问题, 复杂度高

	\vspace{0.60cm}
	\fig{width = 1.00\textwidth}{figs/SI-NPC}
  \end{center}
\end{frame}
%%%%%%%%%%%%%%%%%%%%

%%%%%%%%%%%%%%%%%%%%
\begin{frame}{}
  \begin{center}
	贡献一: 使用 $(\vis, \ar)$ 框架, 为多种 SI 变体提供形式化规约
  \end{center}
\end{frame}
%%%%%%%%%%%%%%%%%%%%

%%%%%%%%%%%%%%%%%%%%
\begin{frame}{}
  \begin{center}
	贡献二: 为 MongoDB 事务一致性协议提供精简的伪代码描述
  \end{center}

  \vspace{0.50cm}
  \begin{columns}
	\column{0.50\textwidth}
	  \fig{width = 0.90\textwidth}{figs/MongoDB-Docs}
	\column{0.50\textwidth}
	  \fig{width = 0.90\textwidth}{figs/MongoDB-Src}
  \end{columns}
\end{frame}
%%%%%%%%%%%%%%%%%%%%

%%%%%%%%%%%%%%%%%%%%
\begin{frame}{}
  \begin{center}
	贡献三: 证明 $\wtalg$、$\rsalg$、$\scalg$ 事务协议
	  分别满足 $\strongsi$、$\rtsi$、$\sessionsi$ 变体
  \end{center}

  \fig{width = 0.80\textwidth}{figs/mongo-txn-si}
\end{frame}
%%%%%%%%%%%%%%%%%%%%

%%%%%%%%%%%%%%%%%%%%
\begin{frame}{}
  \begin{center}
	贡献四: 设计并评估了多项式时间 SI 变体白盒检测算法
  \end{center}

  \fig{width = 0.75\textwidth}{figs/Jepsen-Logo}
\end{frame}
%%%%%%%%%%%%%%%%%%%%