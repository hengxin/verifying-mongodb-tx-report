% preamble.tex

\usepackage{xeCJK}
\usepackage{fontspec}

\usetheme{CambridgeUS} % try Madrid
\usecolortheme{beaver} % try beaver, dolphin, seahorse
% \usefonttheme[onlymath]{serif} % try "professionalfonts"
\usefonttheme{serif}  % standard font (same with that in ``standalone'')
% \setCJKmainfont{Microsoft YaHei} % try SimSun
% \setmainfont{Gill Sans}
% \setsansfont{Gill Sans}

\usepackage[export]{adjustbox}

\setbeamersize{text margin left = 2em, text margin right = 1em}
\setbeamercolor{footnote mark}{fg = teal}
\setbeamertemplate{itemize items}[default]
\setbeamertemplate{enumerate items}[default]

\renewcommand*{\thefootnote}{\alph{footnote}}

\usepackage{graphicx, subcaption}

\usepackage{amssymb, pifont}
\newcommand{\cmark}{\ding{51}}
\newcommand{\xmark}{\ding{55}}
\usepackage{xcolor}

% theorems (global numbering)
\theoremstyle{definition}
\newtheorem{property}[theorem]{Property}
\newtheorem{assumption}{\textsc{Assumption}}

\usepackage{caption}
\DeclareCaptionLabelSeparator{none}{}
\captionsetup{labelsep = none}

\makeatletter
\let\@@magyar@captionfix\relax
\makeatother

\newcommand{\incell}[2]{\begin{tabular}[c]{@{}c@{}}#1\\[3pt] #2\end{tabular}}

% \renewcommand\figurename{图:\;}
% \renewcommand\tablename{表:\;}

% for fig without caption: #1: width/size; #2: fig file
\newcommand{\fig}[2]{
  \begin{figure}[htp]
    \centering
    \includegraphics[#1]{#2}
  \end{figure}
}

% for fig with caption: #1: width/size; #2: fig file; #3: fig caption
\newcommand{\figcap}[3]{
  \begin{figure}[htp]
    \centering
    \includegraphics[#1]{#2}
    \caption{#3}
  \end{figure}
}

\usepackage[linewidth = 1pt, framemethod = TikZ]{mdframed}
\mdfsetup{frametitlealignment=\center}

% for citation
% try "backend = bibtex", "backend = biber"
\usepackage[natbib = true, backend = biber, style = authoryear, maxbibnames = 99]{biblatex}
\setbeamertemplate{bibliography item}[article]
\renewcommand*{\bibfont}{\footnotesize}
\addbibresource{20200930-clientcentric-podc2017.bib}

\newcommand{\ncite}[1]{\violet{\footnotesize [\cite{#1}]}}

% colors
\definecolor{DarkRed}{rgb}{0.55, 0.0, 0.0}
\newcommand{\red}[1]{\textcolor{red}{#1}}
\newcommand{\green}[1]{\textcolor{green}{#1}}
\newcommand{\blue}[1]{\textcolor{blue}{#1}}
\newcommand{\purple}[1]{\textcolor{purple}{#1}}
\newcommand{\cyan}[1]{\textcolor{cyan}{#1}}
\newcommand{\violet}[1]{\textcolor{violet}{#1}}
\newcommand{\lgray}[1]{\textcolor{lightgray}{#1}}
\newcommand{\teal}[1]{\textcolor{teal}{#1}}
\newcommand{\brown}[1]{\textcolor{brown}{#1}}
\newcommand{\orange}[1]{\textcolor{orange}{#1}}

\newcommand{\hl}[2]{\fcolorbox{#1}{#1!60}{#2}}

\newcommand{\thankyou}{
  \begin{frame}[noframenumbering]
    \begin{center}
      \fig{width = 0.618\textwidth}{figs/thankyou}
      Hengfeng Wei (hfwei@nju.edu.cn)
    \end{center}
  \end{frame}
}